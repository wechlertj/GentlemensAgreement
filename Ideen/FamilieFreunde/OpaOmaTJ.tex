In andenken an meinen verstorben Opa und meine im sterben liegende Oma! 

Leider konnte ich meinen Opa nicht so lange miterleben wie ich es wollte, da er starb als ich 6 Jahre alt war. 
Umso mehr weiß ich aus Erzählungen meiner Familie und aus meinen wenigen Erinnerungen, dass er ein großes Vorbild im Leben ist. 
Er war laut meiner Mutter ein sehr umsorgter Mensch der immer für Familie und Freunde da war. Er hat bei finanziellen Probleme aus geholfen, 
war bei einem Streit 
der Schlichter, hat ausgeholfen wenn er konnte und hat sich sehr um den Familienzusammenhalt gekümmert. Viele haben meinen Opa nach Rat gefragt. 

Als mein Opa Krebs hatte und im Krankhaus lag, besuchte ihn eine Schwester. Sie wollte sich bei den besuchen nie verabschieden. 
Er hat ihr gesagt das sie irgendwann Abschied nehmen müssen und sie es machen sollten solange sie es noch können. 
Opa hat die situation so hingenommen wie sie kam und hat nicht getrauert das es zu Ende geht. Es Wurde sehr nüchtern und klar von ihm betrachtet. 
Ich hätte gerne mehr von ihm selbst kennen lernen wollen vorallem in meinem jetzigen Alter das ich viele Fragen an ihn hätte. 

Zu meiner Oma kann ich nur sagen: "WOW". Sie ist zum aktuellen Zeitpunkt in ein Pflegeheim untergekommen da 
sie seit 2 Wochen nicht mehr genug kraft ihren Alltag mit 95 selbst zu beschreiten. 
Vor dieser Zeit hatte sie nich auf ihrem Bauernhof mit meinem Onkel und dessen Familie zusammen gewohnt. 
Es kam einmal am tag eine Pflegekraft die ihr beim waschen half. 
Sie lebte aber sonst noch alleine und war mit ihrer Gehhilfe sehr flick unterwegs (für eine 95 Jährige Dame).
Was sie so besonder für mich macht ist der Umgang mit dem Prozess des alt werdens. 
    - Sie hat sich nie über schmerzen beklagt 
    - Sie stahlt eine große zufriedenheit aus.
    - Sie weiß bis heute von jedem Enkel-/ und Kind was es tut und richtet Grüße aus. 
    - Sie Trauert dem Leben in keinster weiße hinterher
    - Sie will niemanden zur last fallen 
    - Sie freute sich immer über jeden einzelnen der zu ihr kam. 
    - Sie akzeptiert die Situation wie sie ist und macht das beste daraus
    - Sie war selten bis nie krank. (Ich kann mich nur an 3 4 mal in 26 Jahre erinnern)
    - Ihre gesamter Wesenszug und Charakter, die herrangehensweise an Situationen und dei Freude über kleine Dinge
            als meine Früder und ich um weihnahcten mal bei ihr waren, spielten wir Gitarre e-bass und kachon. 
            obwohl wir ein sanftes Rocklied spielten. Sie Freute sich sher über das Leider
    - Sie hatte 11 Enkel und es hab immer zu weihnachten, Geburtstag geld von ihr 

Zusammengefasst ist sie ein absolutes vorbild fürs alt werden. 
Jeder hat mit beiden seine ganz eigene Geschichten erlebt die man weiter erzählen kann und lange in Erinnerung behalten wird. Sie waren Vater, Mutter, Oma, Opa und zum Schluss sogar noch Großmutter.
Sie haben ihren Artbeit mehr als gut gemacht, den aus ihr ist eine Familie gewachsen die es so kein zweites mal auf der Welt gibt. 
Beide haben für die Nachfolgende Generation einen sehr großes Fundament gebaut, von dem man nur provitieren kann. 
Im Andenken an diese zwei Menschen die nicht nur Erinnerungen und Geschichten Hinterlassen sondern auch Familie und Werte hab ich diesen Text geschrieben und versuche ihn noch weiter aus zu bauen. 


Wir verabschieden uns heute von unserer Anneliese. 

Sie ist von uns gagangen aber lebt in jedem von uns weiter. Nicht nur in Erinnerungen die mit der Zeit verblassen mögen. Nein, Viel mehr in der Art und weise wie sie uns alle mit Ihren ganz eigenen geschichten berührt und und vlt sogar verändert hat. 
Angefangen von den Kindern die sie bedingungslos geliebt hat und mit einer selbstverständlichkeit erzogen hat wie man es sich als Kind nur wünschen kann, 
Die Enkelkinder, für die Sie eine warmherzige Oma war und bei der man Kind sein durfte und um Flur fußball spielen konnte, bishin zur Stolzen Uroma, die sich über die neuen in der Familie seh gefreut hat. 

Es ist jetzt die Zeit von trauer und schmerz durch die wir alle müssen. Doch unsere Mutter, Oma und Uroma lässt und nicht alleine damit. Sie hat uns eine Familie geschenkt in der jeder sein Platz hat, ob groß oder klein, jung oder alt.
Wenn gleich auch die zeit des Trauerns ist, ist für mich auch eine Zeit der freude. Sie hatte ein sehr erfühltes und zufriedenstellendes leben. Wir konnten alle bis zum Schluss Teil ihres Lebenswegs sein. 

Sie war und ist ein großes vorbild für mich im bezug auf das lt werden. mit welcher zufriedenheit sie dem ende entgegen geblickt hat sucht ihres gleichen.