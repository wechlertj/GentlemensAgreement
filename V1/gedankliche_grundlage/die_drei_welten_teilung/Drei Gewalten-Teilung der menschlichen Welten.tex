%Jens Gedanken dazu
Die drei Gewalten-Teilung menschlicher Welten

Oft hört ein jener die Worte: Denk daran, die Welt ist dein Spiegel. So etwas zu sagen ist ist leicht, aber auf die Frage warum, kommt eine einfache Antwort.
Sowas in der Art wie: Wer säht wird ernten. Damit ist nicht mehr gesagt als mit der ursprünglichen Aussage: Nichts. So kommt es, dass sich im Zuge der Denkbarkeit ein Muster auszeichnet. 
Wenn die restliche Welt Ihr Spiegel ist, dann sind Sie ebenso
Teil des Spiegels aller anderen. Jeder ist also Teil eines jenen Spiegels aller anderen. Stellen Sie sich vor Sie stehen alleine in einem Raum. Ohne die Möglichkeit einen
Kontakt zu einem anderen Lebewesen herstellen zu können, ist die Welt dann immernoch Ihr Spiegel? Nein. Nun stehen Sie mit einer anderen Person im Raum und sonst nix. Ist die Welt ein Spiegel?
Auch wenn Sie hier instinkiv Ja sagen würden, die Antwort ist nein. Der Grund ist: Der Mensch Ihnen gegenüber ist nicht die Welt, sondern nur Ihr Gegenüber. Nun stehen Sie in einer
Bar mit 50 anderen Leuten. Groß, klein, alt und jung. Ist die Welt jetzt Ihr Spiegel? Schon ehern. Erst in einer Gesellschaft wo Mut (siehe Birkenbihl) eine Rolle spielt entwickelt
sich in einer Gesellschaft die Welt zum Spiegel. Es ist also eine Welt im Spiel, die uns alle gleichermaßen betrifft. Die Frage ist welche. Prinzipiell können in der Welt der Menschen
drei verschiedene Welten existieren.
\begin{enumerate}
    \item \textbf{Realität}\\
    Die Welt, die immer nach den selben Regeln spielt. Eine Welt die allgegenwärtig ist und unabhängig davon wer wir sind und machen existiert. Wir alle sind ein fester Teil von ihr.
    \item \textbf{Fantasie}\\
    Die Welt, die jeweil einem Individuum innewohnt. Völlig losgelöst von anderen und anderem. Unzugänglich für die Außenwelt, außer wir teilen sie mit ihr.
    \item \textbf{Wirklichkeit}\\
    Die Welt, die Watzlawick als die eigene Wirklichkeit beschreibt. Die Welt in der wir aus Bewusstseinssicht leben. Die Welt die sich ein Jeder im Laufe seines Lebens aufbaut.
    Alle unsere eigenen Ansichten geprägt von Erfahrungen, Fähigkeiten und Persönlichkeit. Eine Welt die in jedem von uns einzigartig ist und nur die uns ausmacht, da sie alles 
    darstellt, was wir kennen.
\end{enumerate}









Note:
- Eigene Wirklichkeit
- Die Welt ist ein Spiegel, warum? A.: Realität als gemeinsamer Nenner
