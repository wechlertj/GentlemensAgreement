\chapter{Lebenskunst} 
    \begin{thesis_quotation}

        \chapterquote{Sich gegenseitig begreifen lernen, ist die größte Kunst des Lebens.}{Friedrich Max Müller}{TestBimBo}
    \end{thesis_quotation}
    
    Hier wird im folgenden die Kunst des Lebens beschreiben. 
    {\color{red}Instagram millionäre-mindset Sprüche miteinbringen }
    
    \section{Verhalten}
        \subsection{Erfolg}
            \subsubsection{Erfolg anderer}
                \paragraph{} Der Erfolg anderer ist wert zu schätzen wie der eigen Erfolg. Die 
            \subsubsection{Eigener Erfolg}
                \paragraph{} Den eigenen Erfolg darf man feiern und man darf stolz darauf sein. Was nicht erlaubt ist diesen nutzen und vor anderen damit pralen und angeben. Viel mehr sollte man seinen eigenen erfolg nutzen um dadruch wiederrum anderen zu helfen.
        \subsection{Loyialität}
        \subsection{Pünktlichkeit}
        \subsection{Umsichtigkeit}
        \subsection{Aufrichtigkeit und Ehrlichkeit}
            Aufrichtigkeit: Gesagt getan, Worte und Taten sollte das gleiche Bild abgeben
            Ehrlichkeit: Keine Lügen oer Manipulation gegen andere 
        \subsection{Hilfsbereitschaft}
            Man ist als Gentleman stehts hilfsbereit. Es muss hier aber unterschieden werden ob man mit der Hilfe anderen (dritten) schadet. Falls dies zutrifft so muss man diese Hilfe ausschlagen {\color{red}sutis mike gefängnis}
            wie weit geht man als Gentleman 
        \subsection{Über andere reden}
            wenn man über andere redet wird nie schlecht über eine Person gesprochen, das sie sich selbst nicht verteidigen kann und diese Person in einem schlechtes Licht gerückt wird. Hat diese Person aber andere verletzt auf irgendeine Art und weise ist im Notfall darauf hin zu weisen das andere nicht auch schaden durch diese Person ereilen. \\
            Unter halten sich andere über eine Person negativ entzieht man sich dem Gespräch oder halt sich schlicht aus diesem Thema raus. \\
            Des weiteren fängt man keine lästerei über andere an. 
        \subsection{Wettschulden}  
            Wettschulden sind Ehrenschulden und verliert man eine Wette so hat man dies zu akzeptieren und seine Schuld zu begleichen 
        \subsection{Vertrauen}
            \subsubsection{VErtrauen anderen schenken}
                Einen Gentleman kann man alles anvertrauen da er vertrauliches nicht weiter sagt.
            \subsection{anderen Vertrauen}  
                Man vertraut nicht blind einer aussage oder einer Person (\ref{sec:hinterfragung})
        \subsection{Liebe}
            Liebe ist ein schweres Thema hier wird jedoch versucht dies auf das wesentliche zu beschränken. 
        

    
