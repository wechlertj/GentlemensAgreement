-   Die Frage lautet nicht länger Einzelfall oder Einheitsbrei. Es gibt etwas dazwischen:
    Das Prinzip, das Vielfalt entstehen lässt. (Sebastian Stiller: Der Planet der Algorithmen)

-   Zeichen sind wie der Brennpunkt in meinem Auge: Ein kleiner Punkt, an dem auf der einen 
    Seite der unterschiedlich große Bezugskegel der Bedeutung abgeht. Auf der anderen Seite des Brennpunkts
    öffnet sich etwas: Der Fächer des formalen Schließens. (Sebastian Stiller: Der Planet der Algorithmen)

-   Was du mir sagst, dass vergesse ich.
    Was du mir zeigst, daran erinnere ich mich.
    Was du mich machen lässt, das verstehe ich.
    (Konfuzius ca. 500 v.Chr)