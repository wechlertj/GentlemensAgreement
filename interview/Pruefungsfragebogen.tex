\documentclass[a4paper,12pt]{scrartcl}

\usepackage[utf8]{inputenc}
\usepackage[T1]{fontenc}
\usepackage[ngerman]{babel}

% For the checkbox symbole
\usepackage{amssymb}

\newcommand{\checkbox}{\(\Box\)}
\newcommand{\hfilloutline}[1]{\rule{#1}{0.5pt}}
\newcommand{\frage}[1]{\textit{#1}}

% more space between paragraphs
\setlength{\parskip}{1em}
\setlength{\parindent}{0pt}

\renewcommand{\emph}[1]{\textbf{#1}}

\begin{document}

\title{Prüfungsfragebogen}
\author{der Fachschaft Mathematik}
\date{}
\maketitle
%\vspace{-3em}

\begin{abstract}
Dieser Fragebogen dient dazu, den Studierenden, die nach dir die Prüfung
ablegen wollen, einen Einblick in den Ablauf und den Inhalt zu geben. Das
erleichtert die Vorbereitung und reduziert die Prüfungsangst.  Schreibe bitte
deutlich und verwende einen \emph{schwarzen} Stift, das ist kopierfreundlicher!  
\end{abstract}

\section*{Zur Person}

\frage{Welchen Tätigkeit wird ausgeführt?}
\hfill\hfilloutline{9cm}


\frage{In welchem Bereich?}
\hfill\hfilloutline{9cm}


\frage{Welche Altersgruppe}\\
\checkbox{} unter 20 \hfill
\checkbox{} 20-29 \hfill
\checkbox{} 30-39 \hfill
\checkbox{} 40-49 \\
\checkbox{} 50-59 \hfill
\checkbox{} 60-69 \hfill
\checkbox{} 70-79 \hfill
\checkbox{} 80-89

\frage{Gibt es eine eigene Familie?}\hfill \checkbox{} Ja\hspace{0.8cm}\checkbox{} Nein
\newpage
%\frage{Wer hat die Prüfung abgenommen?}
%\hfill\hfilloutline{7.5cm}


%\frage{Hast du, bist du gerade dabei oder willst du bei einem/einer Prüfenden
%Diplom-/Staats\-examens-/Bachelor-/Master\-arbeit schreiben?}
%\hfill\mbox{\checkbox{} ja\hspace{0.8cm}\checkbox{} nein}


%\frage{Welche Vorlesungen wurden geprüft und bei wem hast du sie gehört?}\\
%\begin{center}
%\vspace{-0.5cm}
%\begin{tabular}{p{0.4\textwidth}|p{0.2\textwidth}|p{0.15\textwidth}|l|l|l}
%	Titel der Vorlesung & Dozent & Wann gehört & SWS & ECTS & LP
%	\\\hline
%	\vspace{1em} &&&&
%	\\\hline
%	\vspace{1em} &&&&
%	\\\hline
%	\vspace{1em} &&&&
%	\\\hline
%	\vspace{1em} &&&&
%	\\\hline
%	\vspace{1em} &&&&
%	\\\hline
%\end{tabular}
%\end{center}
%\newpage


\section*{Fragen zum Thema verbale und nonverbale Kommunikation}

\frage{Was verstehst du unter einer \emph{Disskusion} ?}
\vspace{3cm}

\frage{Was ist für dich ein \emph{Streit}?}
\vspace{3cm}

\frage{Was macht für dich ein \emph{gutes Gespräch} aus?}
\vspace{4cm}


\newpage

   



\section*{Umgang mit Menschen}

\frage{Was ist dir an einer \emph{zwischenmenschlichen} Beziehung am wichtigsten und warum?}
\vspace{4cm}


\frage{Was bedeutet für dich \emph{Respekt}?}
\vspace{3cm}


\frage{Was ist die wichtigste Charaktereigenschaft?} \hfill\hfilloutline{7.5cm}
\frage{Warum?}
\vspace{2cm}



\frage{Wann \emph{vertraust} du einer anderen Person?}
\newpage




\section*{Verhalten der Menschen}

\frage{Wie sollen sich Leute dir gegenüber verhalten?}
\vspace{2cm}


\frage{Würdest du was in der Gesellschaft ändern?}\ \checkbox{} Ja\hspace{0.8cm}\checkbox{} Nein\\
\frage{Wenn ja, was und warum?}
\vspace{3cm}

\subsection*{Gentleman}
\frage{Was verstehst du unter Gentlemen-(schen)(damals)?}
\vspace{2cm} 

\frage{Wie sollte er heute sein?}
\vspace{2cm}

\frage{Was macht ihn aus?}
\vspace{2cm}

\newpage
\frage{Braucht es deiner Meinung nach eine Aufklärung im Umgang mit den Mitmenschen?}\\
\checkbox{} Ja\hspace{0.8cm}\checkbox{} Nein \par
\frage{Wenn ja, warum und worin sollte der Schwerpunkt liegen?}
\vspace{3cm}

\frage{Wie beeinflusst uns das Internet?}
\vspace{4cm}

\frage{Wäre eine Entschleunigung der Gesellschaft deiner Meinung eine gute Maßnahme um Kommuniation untereinander zufördern?}\hfill\checkbox{} Ja\hspace{0.8cm}\checkbox{} Nein

\end{document}


% vim: set ft=tex
