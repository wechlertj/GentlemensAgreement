\chapter{Ansehen / Prestige}
Das Ansehen ist eines der mächtigsten Werkzeuge des Gentleman. Und so wie ein Handwerker seine Werkzeuge pflegt, so ist auch der Gentleman dazu angehalten dem nachzugehen. 

\section{Geburtsrecht}
Es gibt kein Angeborenes Recht auf Ansehen. Ansehen muss aus eigener Kraft erlangt werden.

\section{Zugeständisklausel}
Ansehen kann nicht gehandelt werden und der Mann ist nicht im Stande sich selbst Ansehen zuzugestehen. Das Urteil des Umfeldes macht das Ansehen. Man entscheidet sich dazu ein Gentleman zu sein, den Titel zu tragen muss das Umfeld dem Individuum zugestehen.

\section{Eigennutz}
Ansehen ist eine Macht, mit der große Verantwortung einhergeht. Diese Macht zum Eigennutz zu benutzen ist ein Frevel und inakzeptabel. Wenn es zum Einsatz kommt, dann nur zum Wohle anderer.

\subsection{Erweiterter Eigennutz}
Es ist denkbar das ein Mensch, zu dessen Wohl das Ansehen eines Gentleman eingesetzt wird, die Wohltat ihm/ihr gegenüber nutzt um sich ungerechtfertigt auf Kosten anderer zu bereichern. Es muss darauf geachtet werden, dass dies nicht der Fall ist.

\subsection{Ungerechtfertigte Bereicherung} Eine ungerechtfertigte Bereicherung tritt dann ein, wenn ein Mensch zu Gewinnzwecken in die Irre geführt wird. Rein zur Bereicherung ist dies nicht zulässig.

\subsection{Gerechtfertigte Bereicherung Dritter} Um dem Bösen Einhalt zu gebieten, ist dann eine Irreführung zulässig, wenn sie von der Seite des Irreführenden auf absoluter Ehrlichkeit und Rücksichtnahme, einzig durch verschweigen von Sachverhalten, zustande kommt. Dies auch nur dann, wenn das Wohl mehrere dem Wohl des einzelnen durch diese Tat überwiegt. Auch hier gilt es den Anstand und Manieren beizubehalten.
