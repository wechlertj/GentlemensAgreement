\chapter{Bildung}

    \section{Wissen}
        Auch wenn das Dasein eines Gentleman nicht auf der Tatsache beruht eine schulische oder akademische Bildung vorweisen zu können, so sollte es immer einen Drang geben das Wissen um die Welt zu erweitern. Der Gentleman ist dazu verpflichtet sich in möglichst breitem Maße Wissen anzueignen.
        
        \subsection{Art des Wissens} 
            Die Art des Wissens kann variieren von spezifischem Wissen um auf etwas spezialisiert zu sein, bis hin zu breitem Allgemeinwissen. In welchem Bereich sich Wissen angeeignet wird, kann auf Basis von Stärken und Schwächen entschieden werden.    
        
        \subsection{Von jedem kann man was lernen}
            Vorraussetzung: Er oder sie hat es selbst schon erreicht, erlebt oder durchgeführt.

    \section{Hinterfragung}\label{sec:hinterfragung}
        Eine Aussage, egal von wem sie stammt, darf niemals einfach hingenommen werden ohne sie kritisch zu hinterfragen.

        \subsection{Vorurteil}
            \subsubsection{Allgemeiner Umgang}
                Ein Vorurteil zu bilden ist nicht falsch, es ist sogar nötig. Ohne Vorurteil lässt sich nichts beweisen oder widerlegen. 
                Eben deshalb ist es die Pflicht eines Gentleman darauf zu beharren jedes gestellte Vorurteil zu überprüfen und in diesem Zuge zu beweisen, 
                zu widerlegen, mindestens jedoch mit allen Mitteln dessen bemüht zu sein.

            \subsubsection{Zulässigkeit}
                Das Vorurteil selbst ist dann zulässig, wenn es nicht verallgemeinernd Lebewesen auf einen Materiellen oder Gesellschaftlichen Wert reduziert.

    \section{Bildungspflicht}
        Selbst gebildet zu sein, mindest dessen bemüht, reicht nicht aus um die Welt zu einem besseren Ort zu machen. 
        Es muss weitergegeben werden. 
        Es ist die Pflicht eines Gentleman auf Basis des eigenen Wissens und mentaler zur Verfügung stehender Werkzeuge die Menschen um sich für Andere und Anderes zu sensibilisieren und Wissen (insofern es überprüft ist) weiterzugeben.

        \subsection{Unwissenheit}
            Unwissenheit ist keine Schande an sich. \glqq{} Jeder Mensch hat das Recht auf Lücken\grqq{} (Vera F. Birkenbihl). Es ist dann eine Schande, wenn man wissentlich Unwissenheit verbreitet und als Unwissender nicht bemüht ist der Unwissenheit durch Weiterbildung entgegenzuwirken.

    \section{Jeder ist Lehrer und Lernender}
        Jeder Mensch hat durch eine einzigartige eigene Wirklichkeit einen anderen Blick auf die Welt und ihre Geheimnisse. Gemäß dem was ein anderer an Voraussetzungen vorzuweisen hat ist ein jeder für den Gentleman ein Lehrbuch voller Wissen. Auch hier gilt Abs.~\ref{sec:hinterfragung} der Bildungsklauseln. 
        Ebenso wie der Gentleman angehalten ist aus der Erfahrung anderer an Wissen zu gewinnen, so soll er seine Erfahrungen anderen auf Anfrage hin zur Verfügung stellen.

        \subsection{Stolz} {\color{red} (eher zu Verhalten)}
        Ein Gentleman schweigt und genießt. Niemals darf Stolz das handeln eines Gentleman beeinflussen. Zum entsprechenden Anstand gehört es Erfahrungen und Wissen anderen nicht unter die Nase zu reiben und sich dadurch zu versuchen über andere zu erheben.
