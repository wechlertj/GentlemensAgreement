\chapter*{Vorwort}
%Der Begriff des Gentleman (zu deutsch: Ehrenmann) hat im Laufe der Geschichte viele Wandel durchlaufen. 
%So ist der Ursprung auf den Adel zurückzuführen und dem Bild des idealen rechtschaffenden Mannes. 
%Gott weis der Adel war zum Großteil bemüht dem Titel des Gentleman gerecht zu werden, doch ist seid Ende 
%des 19. Jahrhunderts klar, dass dem nicht zwingend so war. Der Adel ist mittlerweile verschwunden.
% Die kritische Betrachtung des Begriffes, ist auch heute noch vorhanden, lediglich die Hintergründe sind 
% andere. Um einen Konsens zu schaffen, was das Verhalten und die Lebensweise eines modernen Gentleman im 
% klassischen Sinne angeht, so wird mit diesem Dokument der Versuch unternommen einheitliche Regeln und 
% Konzepte festzuhalten. Eines der größten Probleme denen sich dieser Begriff zu stellen hat, 
%sind Vorurteile und der Gender-Mainstream. Ein Gentleman etwa sollte auch heute noch als rechtschaffender, 
%wohlbenehmender, uneigennützig handelnder, gebildeter Mensch sein, der gerade zu pedantisch das Gleichgewicht 
%zwischen Moral und Vernunft zu beherzigen weis.


 %zweiter versuch (gibt noch einen dritten)

 %Ein Gentleman nennt sich nicht als solcher, weil er sich selbst gut behandelt.
 Er nennt sich nicht ein mal als solcher, weil  er andere gut behandelt.
 Er wird als solcher genannt, weil er die ganze Welt gut behandelt.\\ \\

Der Begriff des Gentleman (zu deutsch: Ehrenmann) hat im Laufe der Geschichte viele Wandel durchlaufen. 
So ist der Ursprung auf den Adel zurückzuführen und dem Bild des idealen rechtschaffenden Mannes. 
Gott weis der Adel war zum Großteil bemüht dem Titel des Gentleman gerecht zu werden, doch ist seid ungeraumer Zeit klar, 
dass dem nicht zwingend so war. Damit der Begriff heutzutage nicht mit den Satorialen und dem Hippstartum zusammenfällt,
ebenso der Missbrauch des Wortes und die Wandlung hin zur negativen Konnotation unterbunden wird, ist es nötig ein Leitmotiv bzw. 
einen Leitfaden zur Hand zu haben. Da der Begriff aus Historischer sicht heutzutage zu Recht kritisch betrachtet werden kann, was Shovinismus,
Feminismus und Emanzipation der Frau angeht, stellt sich die Frage wie der Begriff aussehen würde, wenn man ihn heute neu definiert.
Sebastian Stiller sagte in seinem Buch \glqq Planet der Algorithmen\grqq \, dass es oft nicht an dem Willen mangele
fair miteinander umzugehen, sondern lediglich an dem Wissen um die Möglichkeiten dies zu tun.
Ein Gentleman etwa sollte auch heute noch ein rechtschaffender, wohlbenehmender, uneigennützig handelnder, 
gebildeter Mensch sein, der gerade zu pedantisch das Gleichgewicht zwischen Moral und Vernunft zu beherzigen weis.
Wie Sie vielleicht bemerkt haben, haben wir eben bei dem Begriff des Gentleman nicht von einem Mann geredet,
sondern von einem Mensch. Der Grund ist, dass heute nur noch aus biologischen Gründen zwischen Mann und Frau bzw. divers
unterschieden wird. Mann und Frau sind gleichwertig, ebenso divers. Wenn wir uns aber die Knigge angucken (die heutige),
dann stellen wir sehr schnell fest, dass nicht alles daraus modernkonform ist. Es ist nicht so, dass die alte Schule mit
der modernen nicht vereinbar ist, das wirkt lediglich so. Oft fehlen uns die Lösungen solcher Probleme um alt und neu vernünftig
zu vereinen und ein zeitgemäßen Kontext zu schaffen. Hauptsächlich weil die meisten, die sich mit dem Begriff des Gentleman beschäftigen
entweder solange mit dem Gender-Mainstream konfrontiert und kritisiert werden, bis sie aufgeben, oder weil dieser Gender-Mainstream
recht hatte. Was ist aber wenn ich Ihnen sage, nach diesem Buch kann der Begriff des \glqq Modern Gentleman\grqq geführt werden ohne jemanden
oder etwas zu diskriminieren? Was ist wenn ich Ihnen sage, der Begriff ist durchaus auch heute Salongfähig, wir müssen ihn nur modern definieren.
\\
%Zusammenfassend dient ein Gentleman nicht sich selbst, nicht anderen sondern der gesamten Welt und der Menschheit (engl.:\underline{\textbf{Man}}kind).
Wie wir in diesen modernen Zeiten einen gemeinsamen Nenner in Sachen Manieren und Benehmen, aber auch in der Selbstverwirklichung
und Selbstgestaltung zum Wohle der Welt finden, dazu soll Ihnen dieses Buch Hilfestellung liefern.
\\
\\
Mit freundlichen Grüßen,\\ 

Jens Sokat

%dritter Versuch
\glqq Oft mangelt es nicht an dem Willen fair miteinander umzugehen, sondern lediglich an dem Wissen um die Möglichkeiten dies zu tun.\grqq \ So die Aussage von Sebastian Stiller
in seinem Buch \glqq Planet der Algorithmen.\grqq \ Eben um einen gemeinsamen kontext hinsichtlich dem Umgang miteinander zu finden existiert der Name Knigge. Der Begriff stammt
von dem Namen eines Autors namens \glqq Adolph Freiherr von Knigge.\grqq \ Dieser schrieb zu seiner Zeit (DATUM) ein Buch über den Umgang mit Menschen. Heutzutage wird dieser
Begriff als Überbegriff für einen Regelkonsens genutzt, der nicht mehr Zeitgemäß erscheint.
Der Begriff des Gentleman (zu deutsch: Ehrenmann) hat im Laufe der Geschichte viele Wandel durchlaufen. 
So ist der Ursprung auf den Adel zurückzuführen und dem Bild des idealen rechtschaffenden Mannes. 
Gott weis der Adel war zum Großteil bemüht dem Titel des Gentleman gerecht zu werden, doch ist seid ungeraumer Zeit klar, 
dass dem nicht zwingend so war. Damit der Begriff heutzutage nicht mit den Satorialen und dem Hippstartum zusammenfällt,
ebenso der Missbrauch des Wortes und die Wandlung hin zur negativen Konnotation unterbunden wird, ist es nötig ein Leitmotiv bzw. 
einen Leitfaden zur Hand zu haben. Da der Begriff aus Historischer sicht heutzutage zu Recht kritisch betrachtet werden kann, was Shovinismus,
Feminismus und Emanzipation der Frau angeht, stellt sich die Frage wie der Begriff aussehen würde, wenn man ihn heute neu definiert.
Ein Gentleman sieht im Kontext der alten Schule vor ein rechtschaffender, wohlbenehmender, uneigennützig handelnder, 
gebildeter Mann zu sein, der gerade zu pedantisch das Gleichgewicht zwischen Moral und Vernunft zu beherzigen weis.
Im modernen Sinne stellt sich recht schnell die Frage warum nur der Mann und weitergehend gegenüber der Frau?
Gerade im Zuge der Genderdiskussion und der emporsteigenden Stimme nicht gerecht behandelter ist diese Frage mehr als berechtigt.

Der Mensch lebt seither im Rudel (heute nennen wir das Gesellschaft). Schauen Sie sich einmal um, betrachten Sie einmal sich selbst und überlegen Sie was alles in
Ihrem Umfeld nicht vorhanden wäre, wären Sie der einzige Mensch. Schnell fällt auf: Alles wurde von anderen aus der Gesellschaft geleistet. Betrachten wir die Medallie von
der anderen Seite, so sind Sie für jeden anderen aus der Gesellschaft ein Teil der anderen. Eben das macht eine Gesellschaft aus. Eine funktionierende Symbiose aller
Einzelteile. Über den Umgang miteinander gilt es nachzudenken, denn es ist nicht bloß so, dass andere Kleinigkeiten für Sie erledigen, letzten Endes ist Ihr Überleben maßgeblich davon
abhängig, was die anderen der GesellschaftIhnen gegenüber Leisten. Im Zuge der Zweiseitigkeit einer Medallie ebenso umgekehrt.
Nun haben wir uns Verhaltenkodexe im Laufe der Zeit und in verschiedenen Kulturen angeschaut. Alle diese Verhaltensmuster laufen irgendwo in einem gemeinsamen Nenner zusammen.
Da die Welt zunehmend globaler wird haben wir versucht den jeweiligen Kern des Verhaltens jeweils herauszuarbeiten und unternahmen den kühnen Versuch in für den Menschen von Welt
zusammenzufassen. Wir selbst haben eine interessante Entwicklung im Zuge der Rechergen und des Schreibens durchlebt um diesen \glqq Leitfaden \grqq \ für Sie zu verfassen.
Ich würde mich freuen, wenn er Ihnen einen neuen Horizont zu verschaffen vermag. Wenn wir Sie inspirieren können sich Gedanken zu machen, faszinierende Gewohnheiten des Menschen zu entdecken
und vielleicht, wenigstens ein wenig sich dieses Thema zu Herzen nehmen, denn es ist wichtig. Sehr wichtig.

Mit freundlichen Grüßen,
Jens Sokat

%Vorwort von Tim-Jonas Wechler