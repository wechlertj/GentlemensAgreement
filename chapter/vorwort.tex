\chapter*{Vorwort}
%Der Begriff des Gentleman (zu deutsch: Ehrenmann) hat im Laufe der Geschichte viele Wandel durchlaufen. 
%So ist der Ursprung auf den Adel zurückzuführen und dem Bild des idealen rechtschaffenden Mannes. 
%Gott weis der Adel war zum Großteil bemüht dem Titel des Gentleman gerecht zu werden, doch ist seid Ende 
%des 19. Jahrhunderts klar, dass dem nicht zwingend so war. Der Adel ist mittlerweile verschwunden.
% Die kritische Betrachtung des Begriffes, ist auch heute noch vorhanden, lediglich die Hintergründe sind 
% andere. Um einen Konsens zu schaffen, was das Verhalten und die Lebensweise eines modernen Gentleman im 
% klassischen Sinne angeht, so wird mit diesem Dokument der Versuch unternommen einheitliche Regeln und 
% Konzepte festzuhalten. Eines der größten Probleme denen sich dieser Begriff zu stellen hat, 
%sind Vorurteile und der Gender-Mainstream. Ein Gentleman etwa sollte auch heute noch als rechtschaffender, 
%wohlbenehmender, uneigennützig handelnder, gebildeter Mensch sein, der gerade zu pedantisch das Gleichgewicht 
%zwischen Moral und Vernunft zu beherzigen weis.


 %neuversuch

 Ein Gentleman nennt sich nicht als solcher, weil er sich selbst gut behandelt.
 Er nennt sich nicht ein mal als solcher, weil  er andere gut behandelt.
 Er wird als solcher genannt, weil er die ganze Welt gut behandelt.\\ \\

Der Begriff des Gentleman (zu deutsch: Ehrenmann) hat im Laufe der Geschichte viele Wandel durchlaufen. 
So ist der Ursprung auf den Adel zurückzuführen und dem Bild des idealen rechtschaffenden Mannes. 
Gott weis der Adel war zum Großteil bemüht dem Titel des Gentleman gerecht zu werden, doch ist seid ungeraumer Zeit klar, 
dass dem nicht zwingend so war. Damit der Begriff heutzutage nicht mit den Satorialen und dem Hippstartum zusammenfällt,
ebenso der Missbrauch des Wortes und die Wandlung hin zur negativen Konnotation unterbunden wird, ist es nötig ein Leitmotiv bzw. 
einen Leitfaden zur Hand zu haben. Da der Begriff aus Historischer sicht heutzutage zu Recht kritisch betrachtet werden kann, was Shovinismus,
Feminismus und Emanzipation der Frau angeht, stellt sich die Frage wie der Begriff aussehen würde, wenn man heute neu definiert.
Sebastian Stiller sagte in seinem Buch \glqq Planet der Algorithmen\grqq \, dass es oft nicht an dem Willen mangele
fair miteinander umzugehen, sondern lediglich an dem Wissen um die Möglichkeiten dies zu tun.
Ein Gentleman etwa sollte auch heute noch als rechtschaffender, wohlbenehmender, uneigennützig handelnder, 
gebildeter Mensch sein, der gerade zu pedantisch das Gleichgewicht zwischen Moral und Vernunft zu beherzigen weis.
Wie Sie vielleicht bemerkt haben, haben wir eben bei dem Begriff des Gentleman nicht von einem Mann geredet,
sondern von einem Mensch. Der Grund ist, dass heute nur noch aus biologischen Gründen zwischen Mann und Frau bzw. divers
unterschieden wird. Mann und Frau sind gleichwertig, ebenso divers. Wenn wir uns aber die Knigge angucken (die heutige),
dann stellen wir sehr schnell fest, dass nicht alles daraus modernkonform ist. Oft ist es nicht so, dass die alte Schule mit
der modernen nicht vereinbar ist, das wirkt lediglich so. Oft fehlen uns die Lösungen solcher Probleme um alt und neu vernünftig
zu vereinen und ein zeitgemäßen Kontext zu schaffen. Hauptsächlich weil die meisten, die sich mit dem Begriff des Gentleman beschäftigen
entweder solange mit dem Gender-Mainstream konfrontiert und kritisiert werden, bis sie aufgeben, oder weil dieser Gender-Mainstream
recht hatte. Was ist aber wenn ich Ihnen sage, nach diesem Buch kann der Begriff des \glqq Modern Gentleman\grqq geführt werden ohne jemanden
oder etwas zu diskriminieren? Was ist wenn ich Ihnen sage, der Begriff ist durchaus auch heute Salongfähig, wir müssen ihn nur modern definieren.
\\
Zusammenfassend dient ein Gentleman nicht sich selbst, nicht anderen sondern der gesamten Welt und der Menschheit (engl.:\underline{\textbf{Man}}kind).
Wie wir in diesen modernen Zeiten einen gemeinsamen Nenner in Sachen Manieren und benehmen, aber auch in der Selbstverwirklichung
und Selbstgestaltung zum Wohle der Welt finden, dazu soll Ihnen dieses Buch Hilfestellung liefern.
\\
\\
Mit freundlichen Grüßen,\\

Jens Sokat


%Vorwort von Tim-Jonas Wechler