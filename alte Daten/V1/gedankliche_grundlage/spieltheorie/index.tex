
Die Spieltheorie ist eine verknüpfung vieler Ideen und Gedanken. 

Prinzipell geht es darum dass man das Leben als eine Art Spiel sieht. 
Soweit ist dieser Gedanke auch noch nichts neues. von Simon Sinek jedoch kommt dazu ein erweiterter Gedanke. 
Das Leben lässt sich wirklch als Spiel erklären, nicht aber als ein gewöhnliches Spiel was jeder kennt, wie Mensch-Ärgere-Dich-Nicht oder Fußball. 

Diese Spiele haben eines gemeinsam, ein klarer Start, klares Ende und während des Spiel unwiedersetzliche Regeln. Am ende eines Jeden Spiels gibt es EINEN Sieger und einen oder mehrere Verlierer.
Nimmt man jetzt an dass das Leben ein Spiel sei und man versucht diese drei bestandteile darauf zu addaptieren wird man feststellen das es nicht klappt. 
    Haben als sie zur Welt kamen alle zur gleichen Zeit das Spiel begonnen? wohl eher nicht .
    Wissen sie ganz genau was das Leben für einen Sinn, Aufgabe und zweck hat, also die Regeln des Spiels? ziemlich sicher werden sie auch hier keine Antwort parat haben die allgemein für alle gilt. Es ist ja nicht so das die Größten Philosophen der Menschheit sich um diese Frage kümmern 
    Letzte Frage. Wenn sie das Spiel beenden, sei dahin gestellt wie das Lebensspiel bei ihnen zu ende gehen kann, Endet es bei allen anderen zum selben Zeitpunkt. Werden danach Die punkte vergliechen an denen man sich misst und ein Sieger wird erkohren? Auch hier ist es unmöglich Ja zu sagen.

Wie kann man dann das Leben als Spiel sehen und noch davon profitieren? 
Zu beginn habe ich Simon Sinek erwähnt, er ist ein Amerikanischer Schriftsteller
###
 Story Simon Sinek
###

In vielen vorträgen und auch zu letzt in seinem Buch hat er das Prinzip des unendlcihen Spiels erklärt. 
Es ist eigentlich ganz schnell und simpel erklärt, das faszinierende ist wie man das Prinzip im Alltag anwenden kann .... usw.
Die Theorie des unendlichen Spiels besagt 
    - man hat einen freien ein und ausstieg aus dem Spiel. Heißt so viel wie, andere sind schon am spielen und werden nach Ihnen am spielen sein.
    - ihre Anwesenheit beeinflusst das Spiel selbst nicht. 
    - sie haben die freie wahl Regeln zu definieren und zu ändern. Sie sind an kein Regelwerk in erster linie gebunden ( abhängig vom Land in dem man lebt )
    - Das wichtigste das es kein klares Ende gibt, gibt es auch keinen Sieger. 

gehen wir die einzelen Punkte durch 

    Der ein und ausstieg 
        Sie werden geboren, das Spiel ist in voller fahrt und sie lernen jeden Tag mehr über das Spiel während sie es spielen. irgendwann sind sie alt genug und können eigenständig handeln und sich äußern.
        Die Zeit vergeht und wenn alles seinen gewöhnliche wege geht werden sie im ansehnlichen alter sterben und damit aus dem Spiel verschwinden. Nach ihnen geht das Spiel genauso weiter wie es tat bevor sie da waren.

    So mit kommen wir gleich zum nächsten Punkt. Ihre Anwesenheit beeinflusst das Spiel nicht das es einfach weiter vorranschreitet. Wenn man ein Extrem beispiel nimmt (ich betone hier das EXTREM besonders). Wenn sie Hitler oder ähnliche Persönlichkeiten wären verändern sie für eine Zeit das gefüge im Spiel und sind schuld daran das viele vor zeitig das Spiel verlassen aber das Spiel geht erstmal sein gewohnten gang weiter, auch wenn die Spieler zusätzliche herrausfoderungen bevorstehen. 

    Der nächste Punkt war die freie wahl der Regeln.  unter beachtung dass das Grundgesetz eingehalten wird und sie sich ethisch korrekt verhalten. haben sie die Freie Wahl wie das Spiel für sie zu laufen hat. (Bespiele bringen erfolg/entspannt & co.)
        wenn sie für sich ein regelwerk haben und er zu 100% erfühlen haben dann sind sie auf dem besten weg der Sieger zu werden. Der Sieger nach ihren vorstellungen der aber nichts mit dem spiel zu tun hat (formulieren scheiße -> anders machen )

    Da ich den Sieger angesprochen habe, sie können Sieger über sich selbst sein und jedes mal besser werden als das mal davor. sie können ihr bestes geben. 

 