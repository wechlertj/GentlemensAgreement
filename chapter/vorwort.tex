\chapter*{Vorwort}
Der Begriff des Gentleman (zu deutsch: Ehrenmann) hat im Laufe der Geschichte viele Wandel durchlaufen. 
So ist der Ursprung auf den Adel zurückzuführen und dem Bild des idealen rechtschaffenden Mannes. 
Gott weis der Adel war zum Großteil bemüht dem Titel des Gentleman gerecht zu werden, doch ist seid Ende 
des 19. Jahrhunderts klar, dass dem nicht zwingend so war. Der Adel ist mittlerweile verschwunden.
 Die kritische Betrachtung des Begriffes, ist auch heute noch vorhanden, lediglich die Hintergründe sind 
 andere. Um einen Konsens zu schaffen, was das Verhalten und die Lebensweise eines modernen Gentleman im 
 klassischen Sinne angeht, so wird mit diesem Dokument der Versuch unternommen einheitliche Regeln und 
 Konzepte festzuhalten. Eines der größten Probleme denen sich dieser Begriff zu stellen hat, 
 sind Vorurteile und der Gender-Mainstream. Ein Gentleman etwa sollte auch heute noch als rechtschaffender, 
 wohlbenehmender, uneigennützig handelnder, gebildeter Mensch sein, der gerade zu pedantisch das Gleichgewicht 
 zwischen Moral und Vernunft zu beherzigen weis.
