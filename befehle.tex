%Allgemein
    \usepackage{amsmath}
    \usepackage{amsfonts}
    \usepackage{amssymb}
    \usepackage{makeidx}
    \usepackage{graphicx}
    \usepackage{hyperref}
    \usepackage[left=3cm,right=2.5cm,top=2cm,bottom=2cm]{geometry}

%Farben
    \usepackage{xcolor}
    \definecolor{gray}{cmyk}{0,0,0,.8}
    \definecolor{graylight}{gray}{.8}
    \definecolor{reddark}{cmyk}{0,1,.9,.41}
    \colorlet{ctcolortitle}{black}
    \colorlet{ctcolorpartnum}{black}
    \colorlet{ctcolorpartline}{black}
    \colorlet{ctcolorparttext}{black}
    \colorlet{ctcolorchapternum}{black}
    \colorlet{ctcolorchapterline}{gray}
    \colorlet{ctcolorsection}{black}
    \colorlet{ctcolorsubsection}{black}
    \colorlet{ctcolorparagraph}{black}
    
% Schrift
    \usepackage{lmodern}                    % Schriftart (http://www.tug.dk/FontCatalogue/)
    \usepackage[T1]{fontenc}				% Trennung von Umlauten.
	\usepackage[utf8]{inputenc}				% Wird für die direkte Eingabe von Umlauten gebraucht.
	\usepackage[ngerman]{babel}				% Eine Sammlung von verschieden Sprachen, und ermöglicht für diese Sprachen die automatische Worttrennung und die ändert die Bezeichnungen in die jeweilige Sprache. 
											% [ngerman]: Worttrennung nach der neuen Rechtschreibung und deutsche Bezeichnungen. 
    	

    \newcommand{\book}{\fontfamily{pbk}\fontseries{m}\fontsize{11}{13}\selectfont}
    \newcommand{\partlabelfont}{\color{ctcolorpartnum}\nobreak\book\fontsize{80}{80}\selectfont}
    \newcommand{\thesispartfont}{\color{ctcolorparttext}}
    \newcommand{\thesischapterfont}{\color{black}\nobreak\normalfont\huge \fontfamily{phv}\selectfont}
    %\newcommand{\thesissectionfont}{\color{ctcolormain}\nobreak\LARGE\bfseries \tgherosfont}
    \newcommand{\thesissectionfont}{\color{ctcolorsection}\nobreak\normalfont\LARGE \tgherosfont}
    \newcommand{\thesissubsectionfont}{\color{ctcolorsubsection}\nobreak\normalfont\Large \tgherosfont}
    \newcommand{\thesisparagraphfont}{\color{ctcolorparagraph}\nobreak\tgherosfont\small\bfseries}

% > formats: \chapter
    \renewcommand*\chapterheadstartvskip{\vspace*{-8.5em}}
    \renewcommand*\chapterheadendvskip{\vspace*{5\baselineskip}}
    \renewcommand*{\thechapter}{\S\arabic{chapter}}
    \renewcommand*{\chapterformat}{%
		\usekomafont{chapter}%
		{\thechapter\hspace*{10pt}}%
	}
    
% > formats: \section
    \renewcommand*{\thesection}{\arabic{section}}
    \renewcommand*{\sectionformat}{%
		\usekomafont{section}%
		{\thesection\hspace*{5pt}}%
	}

%
% > formats: \subsection

    \renewcommand*{\subsectionformat}{%
        \usekomafont{subsection}%
        {\thesubsection\hspace*{5pt}}%
    }

    \renewcommand*{\subsubsectionformat}{%
        \usekomafont{subsubsection}%
        {\thesubsubsection\hspace*{5pt}}%
    }
% > formats: \parapgraph
    \renewcommand*{\theparagraph}{(\alph{paragraph}.)}
    \renewcommand*{\paragraphformat}{\theparagraph\hspace*{5pt}}
    
    \setcounter{secnumdepth}{\paragraphnumdepth}

% Zitatdarstellung
    \newenvironment{thesis_quotation}%
        {%
            \begin{minipage}{\textwidth}%
                \begin{flushright}
                    \begin{minipage}{.85\textwidth}%
        }%
        {%
                    \end{minipage}%
                \end{flushright}
            \end{minipage}
            \vspace*{1cm}
            %
        }%
    \newcommand{\hugequote}%
        {\book\fontsize{75}{80}\selectfont%
        \hspace*{-.475em}\color{graylight}%
        \textit{\glqq}%
        \vspace{ -.26em}%

    }
    \newcommand{\chapterquote}[3]{%
        \begin{thesis_quotation}%
            \begin{flushleft}
                {\hugequote}\textit{#1}
            \end{flushleft}
            \begin{flushright}
                --- \textbf{#2} \\
                #3
            \end{flushright}
        \end{thesis_quotation}%
    }
    %
    % Clean Quotation environment
    