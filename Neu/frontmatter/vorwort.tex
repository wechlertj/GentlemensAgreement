\chapter{Vorwort}
\textit{\glqq Es gibt wohl keinen Wissensbestand, der nicht einem Zeitgeist verpflichtet ist.\grqq} 
So schreibt es Thomas S. Kuhn in seinem Buch \glqq Die Struktur Naturwissenschaftlicher Revolution.\grqq
Der Begriff des Gentleman hat eine lange Geschichte hinter sich und so wie jeder Begriff, der einen
gesellschaftlichen Kontext führt auch einen Zeitgeist, dem eben jener Begriff verschuldet ist.
War der Begriff des Gentleman im XX Jahrhundert noch ein Französicher Adelstitel, so wandelte er sich Anfang des
20. Jahrhunderts zu der schlichten Bezeichnung anwesender Herrschaften in einer Veranstaltung. Im Zuge
des 21. Jahrhunderts driftet der Begriff zunehmenst in die modischen Kategorien und Bewegungen.
Der Satoriale-Stil, ein aufblühen der 20er Jahre mit moderenen Mitteln und moderner Kleidung.
Das Internet sorgte dafür, dass jeder sich präsentieren konnte und so hat es nicht lange gedauert, dass
die nach Aufmerksamkeit strebenden Männer sich im satorialen Stil zu vermarkten versuchen. Dies hat zur Folge,
dass der Begriff gerade in der Genderdebatte oft eine negative Konotation erhält. Erinnern wir uns an den Ursprung des
Begriffs, so hat dieser eine andere Bedeutung als das gekonnte Einkleiden mit teurer Klamotte.
Der Gentleman in seiner ursprünglichen Form ist ein Adelstitel der jenen zuteil wurde die erstens vom
Adel abstammten und zweitens das Idealbild eines Mannes dieser Zeit verkörperten. Der Gentilhomme, so der
französiche Begriff, war also ein rechtschaffender, sozialer, uneigennützig denkend und handelnder Mann.
In der Moderne wird dieser Begriff keines Wegs mehr auf diese Art verwendet. Ein einfacher Blick in die
\glqq sozialen \grqq Medien und der Gentleman als Suchbegriff, veranschaulicht schnell, worum der Begriff sich
in dieser Welt dreht. Es stellt sich die Frage, wie dieser Begriff zu dem Bild gelangte, dass er heutzutage vertritt.
In einer Welt, in der sich jeder Mensch auf jede erdenkliche Art und Weise jedem anderen präsentieren kann, ist
der Versuch außergewöhnlich zu sein ein schwieriges Unterfangen. Die Gruppen der Bedrachter werden größer und somit
sinkt die Wahrscheinlichkeit sich aus der Masse hervorzuheben. Einzigartigkeit ist die normalste Eigenschaft der Welt.
Jedes noch so kleine Elementarteilchen ist durch die Tatsache, dass es sich an einem anderen Ort befindet als alle anderen,
schon einzigartig. Jedes daraus zusammengesetzte System ebenso. Nach Einzigartigkeit zu streben ist demnach vergleichbar damit,
auf eine Ziellinie zulaufen zu wollen, die man bereits hinter sich gelassen hat. Deswegen muss als Gentleman der Netzwelt
ein Mittel herangezogen werden, dass die eigentlichen Werte nicht vertritt. Es wird präsentiert, dass man etwas hat,
dass alle anderen wollen. Diese Gentleman handelt eigennützig und denken in erster Linie an ihr Ansehen.
Bedauerlicher Weise ist dies ein Streben nach Einzigartigkeit, dass wie wir mittlerweile wissen zu nichts führt, außer einem
ewigen Kreislauf des Strebens nach dem was andere wollen. Bekannt ist, wer schon hat, bekommt noch mehr.
Wer also das Ansehen in die Netzwelt hat, hat die Möglichkeit sich durch diese Eigenschaft noch mehr von dem
anzueignen, was andere wollen. In der Regel ist es Geld. Geld befreit einen von Sorgen so heisst es. Aussehen, denn 
wer gut aussieht, hat weniger Konkurrenz und kann sich besser verkaufen. Schenkt man diesen Menschen die Aufmerksamkeit
die sie anstreben, dann weil diese etwas haben was wir wollen. Da sie aber eben genau das haben was wir wollen und 
sie durch das höhere Maß an Aufmerksamkeit, damit Ansehen und damit Macht, immer mehr von dem anhäufen was wir wollen 
werden Menschen die die Aufmerksamkeit schenken dadurch immer ärmer. Wir schenken diesen Menschen das, wonach wir eigentlich
streben. Wir wollen beachtet und geschätzt werden für das, was wir tun. Das Streben danach basiert nicht auf dem
Wunsch einzigartig zu sein, sondern außergewöhnlich und das ist eine andere Sache. Es gilt anzufangen daran zu denken
woher die Beachtung unserer selbst kommt und die Wechselwirkungen, die sie mit sich führt.

An dieser Stelle kommt ein Mann ins Spiel
der das Problem zur Epoche der Aufklärung erkannte und in Worte gefasst hat.