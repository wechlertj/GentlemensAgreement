\chapter{Memento Mori führt zu Carpe Diem}
%Die endlichkeit des Lebens schenkt uns die Begrenzung des Lebens. Ohne die Begrenzung des lebens würde eine Ausgestaltung kein Sinn ergeben.
Haben Sie sich schon mal gefragt wie viel Zeit Sie am Tag verbrauchen um sich zu ärgern?\par
Sicherlich nicht. Ich habe es auch nicht, bis ich einen Betrag von René Bornonus gesehen habe. In d







Denken Sie an das Sterben, daran das Ihre Zeit bald abgelaufen ist? Das Leben ist für jeden einzelnen endlich. 
Wenn man heranwächst und sein Leben lebt möchte man irgendwann gewisse Ziele erreicht haben. Wenn man sich keine Gedanken über die Endlichkeit des eigenen Lebens macht schiebt man die Ziele auf.
Klassiker sind hier die Neujahresvorsätze die spätestens zum März bei den meisten wieder aufgeschoben werden. \par\medskip
Wir haben nur begrenzt Zeit. Von Minute eins in unserem Leben läuft die Zeit die uns bleibt Rückwärts. Jede Minute, jede Stunde, jeder Tag der verstrichen ist können wir nicht zurückholen. 
Und Nein, Zeitreisen wird es nicht geben, nicht heute und auch in der Zukunft nicht. Es ist physiklaisch schlicht unmöglich. Man muss sich im Klaren sein, dass jeder Moment unwiederruflich vorrüber geht.\par\medskip
Sie fragen sich jetzt sicher, warum ich das Ihnen so ausführlich nieder schreiben? Durch die Begrenzung der Zeit die jedem einzelnen zur verfügung stehet, haben wir den Zwang uns Zeitnah mit unseren Zielen auseinander zu setzen.
Es kommt die Zeit in der Ihr Leben zu Ende geht, das ihrer geliebten und derer die Sie nicht mögen. Wäre es nicht schöner diese begrenzte Zeit mit sinnvollen, schönen  Dingen zu füllen, statt alles vor sich hinzuschieben. \par\medskip
\section{Memento Mori}

\section{Carpe Diem}